\chapter{Срединный путь}

Учение Буддизма состоит в развитии доброты и отказе от зла. Когда же доброта будет развита, а злу будет отказано, следует отпустить и добро и зло. Мы уже говорили достаточно о добре и зле, чтобы что-то в этом смыслить, поэтому я бы хотел поговорить о Срединном пути, превосходящем и то, и другое.

Цель разговоров о Дхамме и учении Будды – указать способ избавления от страданий тем, кто их испытывает. Цель же самого учения состоит в объяснении правильного понимания. Неверное понимание не позволит достичь умиротворения.

Обретшие просветление Будды в своих первых лекциях указывали на две крайности, приводящие к перерождениям – потакание страстям и крайний аскетизм. Колеблясь меж этими полярными одержимостями невозможно достичь умиротворения.

Просветленный видел как все живые существа мечутся между крайностями, не зная об их недостатках и не ведая о  Срединном пути. Так как мы одни из них, так как мы все еще испытываем желания, мы лишены свободы. Будда объяснял, что подобное опьянение должно быть чуждо медитирующему. Потокание страстям и крайний аскетизм или, говоря проще, безволие и жесткость не приводят к миру.

Внимательное наблюдение за собой позволит увидеть, что жесткость и напряженность проистекают из гнева и приводят к скорби и горю. Потакание страстям приводит к сиюминутному счастью. Оба состояния – счастья и горя – отличны от умиротворения. Будда учил отпустить оба этих состояния. Это и есть верная практика. Это Срединный путь.
