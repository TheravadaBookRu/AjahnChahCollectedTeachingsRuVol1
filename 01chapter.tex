\chapter{Срединный путь}

Учение Буддизма состоит в развитии доброты и отказе от зла. Когда же доброта будет развита, а злу будет отказано, следует отпустить и добро и зло. Мы уже говорили достаточно о добре и зле, чтобы что-то в этом смыслить, поэтому я бы хотел поговорить о Срединном пути, превосходящем и то, и другое.

Цель разговоров о Дхамме и учении Будды – указать способ избавления от страданий тем, кто их испытывает. Цель же самого учения состоит в объяснении правильного понимания. Неверное понимание не позволит достичь умиротворения.

Обретшие просветление Будды в своих первых лекциях указывали на две крайности, приводящие к перерождениям – потакание страстям и крайний аскетизм. Колеблясь меж этими полярными одержимостями невозможно достичь умиротворения.

Просветленный видел как все живые существа мечутся между крайностями, не зная об их недостатках и не ведая о  Срединном пути. Так как мы одни из них, так как мы все еще испытываем желания, мы лишены свободы. Будда объяснял, что подобное опьянение должно быть чуждо медитирующему. Потокание страстям и крайний аскетизм или, говоря проще, безволие и жесткость не приводят к миру.

Внимательное наблюдение за собой позволит увидеть, что жесткость и напряженность проистекают из гнева и приводят к скорби и горю. Потакание страстям приводит к сиюминутному счастью. Оба состояния – счастья и горя – отличны от умиротворения. Будда учил отпустить оба этих состояния. Это и есть верная практика. Это Срединный путь.

Слова «Срединный путь» относятся к уму, а не к телу или речи. Наш ум приходит в замешательство при соприкосновении с неприятными впечатлениями и это не Срединный путь. Наш ум потакает приятному когда приходит в соприкосновении с тем, что нам нравится и это также не является Срединным путем. 

Мы, люди, не желаем страданий, мы желаем счастья. На деле же счастье — это изощренная форма страданий. Счастье можно сравнить с хвостом змеи, а горе с её головой. Голова змеи очень опасна своими ядовитыми клыками: стоит её коснуться и она немедленно вас укусит. Не менее опасен и хвост змеи: ухватитесь за него и змея, развернувшись, снова укусит вас, так как и голова и хвост принадлежат одной и той же змее.

Точно так же счастье и горе, наслаждение и отвращение происходят от одного источника — желания. Когда мы счастливы наш ум в действительности не умиротворен. Например, мы обретаем желаемое богатство, престиж, похвалу или удовольствие — мы довольны. Тем не менее ум становится беспокойным, опасаясь потери. Такой затаенный страх обернется страданиями позже, когда мы в конце концов потерям обретенное.

Поэтому если не быть осознанным, то даже счастье приведет к неминуемым страданиям. Это все равно что хвататься за хвост змеи: она нас укусит, если её не отпустить. Будь это хвост змеи или её голова, благотворные условия или нежелательные — все это свойства «Колеса Бытия», бесконечных перемен.

Будда провозгласил нравственность, сосредоточенность и мудрость путём к умиротворению и просветлению. Перечисленные качества — это просто путь и они не являются сутью Буддизма. Будда называл их magga, что буквально означает «путь». Сущность же Буддизма в умиротворении, которое проистекает из объективного понимания природы вещей. При детальном рассмотрении становится очевидно, что умиротворение отлично от счастья и горя. Ни то, ни другое не являются объективной истиной.

Человеческий ум — объект познания, который Будда увещевал исследовать. Он являет собой нечто, что проявляет себя в деятельности, без которой ум невозможно оценить и познать. В своем естественном состоянии ум непоколебим и неподвижен. Когда же возникает ощущение счастья ум теряется и блекнет на фоне новых впечатлений. Такое поведение ума приводит к возникновению привязанности.

Будда уже разработал целостную практику для осознания природы вещей, о которой мы чаще говорим, чем применяем в жизни. Наш ум и наша речь все еще не гармоничны и мы часто придаемся празднословию. Разговоры и попытки угадать истину об окружающей действительности не увенчаются успехом. Даже если об этой истине говорить прямо она не будет услышана тем, кто не практикует. С другой стороны, тот, кто постиг природу вещей больше не нуждается в наставлениях. Вот почему Будда говорил: «Просветленный только указывает путь». Никто не может за вас начать практику. Никто не может сформулировать истину словами и рассказать о ней.

Все подбные наставления, несмотря на различия, призваны приблизить ум к пониманию истины. Без понимания истины мы обречены страдать. К примеру, мы часто используем термин «sa\.nkh\={a}r\={a}», когда говорим о теле. Любой может повторить это слово. На деле же мы сталкиваемся с трудностями просто потому что мы не знаем природы  sa\.nkh\={a}r\={a}. Не зная природы тела мы страдаем.

Вот пример. Представьте, что однажды утром вы идете на работу и   слышите грубые оскорбления в ваш адрес от мужчины по другую сторону улицы. Стоит вам услышать брань в свой адрес и ум перестает быть таким как прежде: вы больше не чувствуете себя хорошо, вас переполняет гнев и обида, а оскорбитель преследует вас день и ночь, не давая вам покоя даже дома и вы хотите с ним расквитаться.

Спустя несколько дней ваш знакомый говорит: «Расслабься! Тот человек, что оскорбил тебя тогда — он не в себе, он сумасшедший! Вот уже несколько лет! Он поносит всех кого видит. Никто не принимает его всерьез». Стоит вам это услышать и вы немедленно почувствуете облегчение. Злоба и обида, которые гложили вас расстаяли без следа. Почему? Потому что вы знаете в чем дело. До этого вы злились, думая, что вас оскорбил нормальный человек и такое ошибочное понимание приносило страдания. Но стоило узнать реальное положение дел и все изменилось: «Он же просто не в себе! Это все объясняет!»

Когда вы поняли всю ситуацию вы почувствовали себя отлично. Объективный взгляд разрешает проблемы. Наоборот, ошибочное представление приводит к привязанности. Когда вы думали, что тот человек, что оскорблял вас нормален вы были готововы его убить. Но узнав, что он сумасшедший все разрешилось само собой. Такова сила истины.

Те, кто видят Дхамму имеют подобный опыт. Привязанность, отвращение и заблуждение исчезают также легко при правильном понимании. Когда верного понимания нет мы думаем: «Жадность и неприятие переполняют меня. Что я могу поделать?» Это всего лишь свидетельствует об отсутствии ясного понимания, как в примере с сумасшедшим: знание о том, что тот человек болен полностью избавляет от беспокойств. Рассказы о том, как хорошо не испытывать беспокойств не помогут вам. Ключ к разрешению проблем — это ясное знание. Оно выкорчевывает и избавляет от привязанности.

Метод одинаков и для тела (sa\.nkh\={a}r\={a}). Хотя Будда уже объяснил, что тело не является нашей истинной сутью мы не согласны с этим и продолжаем упорно за него хвататься. Если бы тело могло говорить, оно бы постоянно сообщало нам: «Ты мне не хозяин». И в реальности оно сообщает нам это постоянно, но на языке Дхаммы. И это наша проблема, что мы этот язык не понимаем.

Органы чувств — глаза, уши, нос, язык, кожа — постоянно изменяются. И я не припомню и раза, чтобы эти органы испросили у меня на это разрешения! Когда появляется головная боль или боль в животе тело не спрашивает на это разрешения. Оно просто следует своему естественному курсу. Все это лишь подтверждает, что если у тела и есть хозяин, то это точно не вы. Будда описывал тело как объект без сущности.

Не понимая Дхаммы мы не понимаем и sa\.nkh\={a}r\={a}, привязываясь к нему, считая его чем-то, что принадлежит нам. Такая привязанность приводит к становлению\footnote{(Pali: paticcasamupp\={a}da) взаимозависимое возникновение. Становление (bhava) — одно из 12 звеньев взаимозависимого возникновения.} и, как следствие, к рождению. Рождение — причина старости, болезни, смерти и массы всевозможных страданий.

Мы коснулись pa\d{t}iccasamupp\={a}da, согласно которой невежество (avijj\={a}) порождает волевые наклонности (sa\.nkh\={a}r\={a}), которые порождают сознание (vij\~{n}\={a}na) и так далее. Все это происходит только в нашем уме. Когда наши чувства соприкасаются с чем-то неприятным — и если это происходит без внимания с нашей стороны — возникает невежество и тотчас возникают страдания. Наш ум перетерпевает все эти изменения так быстро, что мы едва ли сможем за ним поспеть. Это похоже на падение с дерева. Немножко отвелкся — бац! — и ты шлепнулся о землю. В действительности ты пролетал мимо веток, но ты не считал их и едва ли их запомнил — все происходило слишком быстро. Ты просто падал, а потом шлепнулся.

Также и  pa\d{t}iccasamupp\={a}da. Палийский канон объясняет двенадцатичленную формулу зависмого возникновения так: невежество порождает волевые наклонности, волевые наклонности порождают сознание, сознание порождает имя и форму (n\={a}mar\={u}pa), имя и форма порождают шесть чувственных опор (sa\d{l}\={a}yatana), шесть чувственных опор порождают контакт чувств с объектом (phassa), контакт чувств с объектом порождает приятные или неприятные или нейтральные ощущения (vedan\={a}), чувства порождают жажду (ta\d{n}ha), жажда порождает привязанность (up\={a}d\={a}na), привязанность порождает становление (bhava), становление порождает рождение (j\={a}ti), рождение порождает взросление, увядание и смерть (jar\={a}-mara\d{n}a). В действительности, когда мы соприкасаемся с чем-то неприятным мы испытываем страдания немедленно! Это мгновенное страдание также является результатом изменений ума в соответсвии с  pa\d{t}iccasamupp\={a}da. Вот почему Будда часто объяснял ученикам важность глубокого понимания процессов, происходящих в уме.

Люди рождаются в этом мире без имен – имя они получают позже. Таков обычай. Мы даем имена для удобства, чтобы знать как к кому обращаться. Писания делают то же: они разделяют изучаемый предмет и дают каждой части имя для удобства изучения действительности. Так, всё есть просто sa\.nkh\={a}r\={a}. Все в этом мире может быть рассмотрено как совокупность вещей. Будда говорил, что все в этом мире временно, не приносит удовлетворения и не имеет отношения к нам. Мы не понимаем это твёрдо и как следствие имеем ложное представление о природе вещей: мы думаем, что sa\.nkh\={a}r\={a} принадлежит нам или мы и есть sa\.nkh\={a}r\={a} или, что счастье и горе это что-то наше собственное или мы думаем, что мы счастливы или мы страдаем. Подобные представления не являются полными и ясными относительно истинной природы вещей. Истина заключается в том, что мы не можем заставить вещи следовать нашим желаниям — они следуют своим природным курсом.

Предположим, ты вышел и сел посреди автобана, забитого машинами или грузовиками. Будет глупо злиться и кричать: “Перестаньте здесь ездить!” Это автобан и будет несправедливо требовать этого. Что же тогда делать? Нужно уйти с дороги! Дорога предназначена для движения транспорта. В противном случае, желание прекратить движение на автобане будет доставлять страдания.

