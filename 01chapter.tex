\chapter{Срединный путь}

Учение Буддизма состоит в развитии доброты и отказе от зла. Когда же доброта будет развита, а злу будет отказано, следует отпустить и добро и зло. Мы уже говорили достаточно о добре и зле, чтобы что-то в этом смыслить, поэтому я бы хотел поговорить о Срединном пути, превосходящем и то, и другое.

Цель разговоров о Дхамме и учении Будды – указать способ избавления от страданий тем, кто их испытывает. Цель же самого учения состоит в объяснении правильного понимания. Неверное понимание не позволит достичь умиротворения.

Обретшие просветление Будды в своих первых лекциях указывали на две крайности, приводящие к перерождениям – потакание страстям и крайний аскетизм. Колеблясь меж этими полярными одержимостями невозможно достичь умиротворения.

Просветленный видел как все живые существа мечутся между крайностями, не зная об их недостатках и не ведая о  Срединном пути. Так как мы одни из них, так как мы все еще испытываем желания, мы лишены свободы. Будда объяснял, что подобное опьянение должно быть чуждо медитирующему. Потокание страстям и крайний аскетизм или, говоря проще, безволие и жесткость не приводят к миру.

Внимательное наблюдение за собой позволит увидеть, что жесткость и напряженность проистекают из гнева и приводят к скорби и горю. Потакание страстям приводит к сиюминутному счастью. Оба состояния – счастья и горя – отличны от умиротворения. Будда учил отпустить оба этих состояния. Это и есть верная практика. Это Срединный путь.

Слова «Срединный путь» относятся к уму, а не к телу или речи. Наш ум приходит в замешательство при соприкосновении с неприятными впечатлениями и это не Срединный путь. Наш ум потакает приятному когда приходит в соприкосновении с тем, что нам нравится и это также не является Срединным путем. 

Мы, люди, не желаем страданий, мы желаем счастья. На деле же счастье — это изощренная форма страданий. Счастье можно сравнить с хвостом змеи, а горе с её головой. Голова змеи очень опасна своими ядовитыми клыками: стоит её коснуться и она немедленно вас укусит. Не менее опасен и хвост змеи: ухватитесь за него и змея, развернувшись, снова укусит вас, так как и голова и хвост принадлежат одной и той же змее.

Точно так же счастье и горе, наслаждение и отвращение происходят от одного источника — желания. Когда мы счастливы наш ум в действительности не умиротворен. Например, мы обретаем желаемое богатство, престиж, похвалу или удовольствие — мы довольны. Тем не менее ум становится беспокойным, опасаясь потери. Такой затаенный страх обернется страданиями позже, когда мы в конце концов потерям обретенное.

Поэтому если не быть осознанным, то даже счастье приведет к неминуемым страданиям. Это все равно что хвататься за хвост змеи: она нас укусит, если её не отпустить. Будь это хвост змеи или её голова, благотворные условия или нежелательные — все это свойства «Колеса Бытия», бесконечных перемен.

Будда провозгласил нравственность, сосредоточенность и мудрость путём к умиротворению и просветлению. Перечисленные качества — это просто путь и они не являются сутью Буддизма. Будда называл их magga, что буквально означает «путь». Сущность же Буддизма в умиротворении, которое проистекает из объективного понимания природы вещей. При детальном рассмотрении становится очевидно, что умиротворение отлично от счастья и горя. Ни то, ни другое не являются объективной истиной.

Человеческий ум — объект познания, который Будда увещевал исследовать. Он являет собой нечто, что проявляет себя в деятельности, без которой ум невозможно оценить и познать. В своем естественном состоянии ум непоколебим и неподвижен. Когда же возникает ощущение счастья ум теряется и блекнет на фоне новых впечатлений. Такое поведение ума приводит к возникновению привязанности.

Будда уже разработал целостную практику для осознания природы вещей, о которой мы чаще говорим, чем применяем в жизни. Наш ум и наша речь все еще не гармоничны и мы часто придаемся празднословию. Разговоры и попытки угадать истину об окружающей действительности не увенчаются успехом. Даже если об этой истине говорить прямо она не будет услышана тем, кто не практикует. С другой стороны, тот, кто постиг природу вещей больше не нуждается в наставлениях. Вот почему Будда говорил: «Просветленный только указывает путь». Никто не может за вас начать практику. Никто не может сформулировать истину словами и рассказать о ней.

Все подбные наставления, несмотря на различия, призваны приблизить ум к пониманию истины. Без понимания истины мы обречены страдать. К примеру, мы часто используем термин «sa\.nkh\={a}r\={a}», когда говорим о теле. Любой может повторить это слово. На деле же мы сталкиваемся с трудностями просто потому что мы не знаем природы  sa\.nkh\={a}r\={a}. Не зная природы тела мы страдаем.

Вот пример. Представьте, что однажды утром вы идете на работу и   слышите грубые оскорбления в ваш адрес от мужчины по другую сторону улицы. Стоит вам услышать брань в свой адрес и ум перестает быть таким как прежде: вы больше не чувствуете себя хорошо, вас переполняет гнев и обида, а оскорбитель преследует вас день и ночь, не давая вам покоя даже дома и вы хотите с ним расквитаться.

Спустя несколько дней ваш знакомый говорит: «Расслабься! Тот человек, что оскорбил тебя тогда — он не в себе, он сумасшедший! Вот уже несколько лет! Он поносит всех кого видит. Никто не принимает его всерьез». Стоит вам это услышать и вы немедленно почувствуете облегчение. Злоба и обида, которые гложили вас расстаяли без следа. Почему? Потому что вы знаете в чем дело. До этого вы злились, думая, что вас оскорбил нормальный человек и такое ошибочное понимание приносило страдания. Но стоило узнать реальное положение дел и все изменилось: «Он же просто не в себе! Это все объясняет!»

Когда вы поняли всю ситуацию вы почувствовали себя отлично. Объективный взгляд разрешает проблемы. Наоборот, ошибочное представление приводит к привязанности. Когда вы думали, что тот человек, что оскорблял вас нормален вы были готововы его убить. Но узнав, что он сумасшедший все разрешилось само собой. Такова сила истины.

Те, кто видят Дхамму имеют подобный опыт. Когда привязанность, отвращение и заблуждение исчезают, то происходит это также легко.

